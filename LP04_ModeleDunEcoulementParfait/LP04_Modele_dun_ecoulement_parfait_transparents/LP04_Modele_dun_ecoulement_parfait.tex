% IMPORTANT: PLEASE USE XeLaTeX FOR TYPESETTING
\documentclass[10pt]{beamer}

\usetheme{Darmstadt}%{default}
\usecolortheme{beaver}
\usepackage[T1]{fontenc} 
\usepackage[utf8]{inputenc}
\usepackage[french]{babel}
\usefonttheme{serif}
\usepackage{lmodern}
\usepackage{tcolorbox}
 % pour un pdf lisible à l'écran
 % il y a d'autres choix possibles 
\usepackage{pslatex}
% \usepackage{ctex, hyperref}
\usepackage{latexsym,amsmath,xcolor,multicol,booktabs,calligra}
\usepackage{graphicx,pstricks,listings,stackengine}
\usepackage{chemfig}

\usepackage{tabularx}
% meta-data
\title{Leçon :Effet tunnel, radioactivité $\alpha$}

\author{Gabriel Le Doudic}
\institute{Préparation à l'agrégation de Rennes}
% \titlebackground{images/background}

\definecolor{aquamarine}{rgb}{0.5, 1.0, 0.83}
\definecolor{applegreen}{rgb}{0.55, 0.71, 0.0}	
\definecolor{cobalt}{rgb}{0.0, 0.28, 0.67}

\definecolor{definitionf}{RGB}{220,252,220}
\definecolor{definitionl}{RGB}{39,123,69}
\definecolor{definitiono}{RGB}{72,148,101}

\definecolor{propositionf}{RGB}{255,216,218}
\definecolor{propositionl}{RGB}{38,38,38}
\definecolor{propositiono}{RGB}{109,109,109}

\definecolor{theof}{RGB}{255,216,218}
\definecolor{theol}{RGB}{160,0,4}
\definecolor{theoo}{RGB}{221,65,100}

\definecolor{avertl}{RGB}{163,92,0}
\definecolor{averto}{RGB}{255,144,0}

\definecolor{histf}{RGB}{241,238,193}

\definecolor{metf}{RGB}{220,230,240}
\definecolor{metl}{RGB}{56,110,165}
\definecolor{meto}{RGB}{109,109,109}


\definecolor{remf}{RGB}{230,240,250}
\definecolor{remo}{RGB}{150,150,150}

\definecolor{exef}{RGB}{240,240,240}

\definecolor{protf}{RGB}{247,228,255}
\definecolor{protl}{RGB}{105,0,203}
\definecolor{proto}{RGB}{174,88,255}

\definecolor{grid}{RGB}{180,180,180}

\definecolor{titref}{RGB}{230,230,230}

\definecolor{vert}{RGB}{23,200,23}

\definecolor{violet}{RGB}{180,0,200}

\definecolor{copper}{RGB}{217, 144, 88}
%% CADRES

\newtcolorbox{defi}[1]{
	colback=applegreen!5!white,
  	colframe=applegreen!65!black,
	fonttitle=\bfseries,
  	title={#1}}
\newtcolorbox{Programme}[1]{
	colback=cobalt!5!white,
  	colframe=cobalt!65!black,
	fonttitle=\bfseries,
  	title={#1}}  
\newtcolorbox{Resultat}[1]{
	colback=theof,%!5!white,
	colframe=theoo!85!black,
  fonttitle=\bfseries,
	title={#1}} 
\usepackage{tikz}
\usepackage{array}
\usepackage[scientific-notation=true]{siunitx}
\usetikzlibrary{matrix}
\newcommand{\diff}{\mathrm{d}}

\title{Leçon : Modele d'un écoulement parfait}

% document body
\begin{document}
\begin{frame}{}
    \titlepage

    \begin{tabularx}{\textwidth}{l@{:\,\,}X}
        \textbf{Niveau} 	  & CPGE\\
        \textbf{Prérequis} & Cinématique des fluides\\
        & Hydrostatique \\
        & 			Mécanique
    \end{tabularx}
\end{frame}

\begin{frame}
    \tableofcontents
\end{frame}

\section{Modèle de l'écoulement parfait}
\subsection{Définition}
\subsection{Équation d'Euler}
\begin{frame}{Conditions aux limites}
    \begin{table}
      \centering
      \resizebox{\columnwidth}{!}{
      \begin{tabular}{|c|c|c|}
        \hline
        Type d'interface & Écoulement parfait (Euler) & Écoulement visqueux (Navier Stokes) \\ \hline 
        Paroi solide & $\begin{array}{ll} \left. v_\perp \right|_{\rm fluide} & = \left. v_\perp \right|_{\rm paroi}\\ P_{\rm fluide} & = P_{\rm paroi} \end{array}$& $ \begin{array}{ll} \vec{v}_{\rm fluide} & = \vec{v}_{\rm paroi};\\ P_{\rm fluide} &= P_{\rm paroi}\\ +& \text{équation sur} \text{la contrainte tangentielle} \end{array}$ \\ \hline
        $\begin{array}{ll}\text{Interface fluide sans}\\ \text{tension de surface}\end{array}$ & $\begin{array}{ll} \left. v_\perp \right|_{\rm 2} & = \left. v_\perp \right|_{\rm 2}\\ P_{1} & = P_{2} \end{array}$ & $ \begin{array}{ll} \vec{v}_{1} & = \vec{v}_{2};\\ P_{1} &= P_{2}\\ +& \text{équation sur} \text{la contrainte tangentielle} \end{array}$ \\ \hline
        $\begin{array}{ll}\text{Interface fluide avec}\\ \text{tension de surface}\end{array}$ & $\begin{array}{ll} \left. v_\perp \right|_{\rm 2} & = \left. v_\perp \right|_{\rm 2}\\ P_{1} -P_{2}& =\gamma\left(\frac{1}{R}+\frac{1}{R^\prime}\right) \end{array}$ & $ \begin{array}{ll} \vec{v}_{1} & = \vec{v}_{2};\\ +& \text{équation sur} \text{la contrainte tangentielle} \end{array}$ \\ \hline
      \end{tabular}  
      
      }
    \end{table}
  \end{frame}

\subsection{Conséquences}

\begin{frame}{\insertsubsection}
    \begin{figure}
        \centering
        % \includegraphics[options]{name}
        \caption{Illustration d'un écoulement parfait autour d'une boule (Hydrodynamique Physique Guyon HUlin Petit)}
    \end{figure}
    On peut également illustrer à l'aide de cette vidéo : \href{https://www.youtube.com/watch?v=2Z6UJbwxBZI&ab_channel=ryanhaart}{ici}
\end{frame}

\section{Théorême de Bernouilli}
\subsection{énoncé}
\subsection{Application}
\section{Limite du modèle: écoulement réel}
\begin{frame}{\insertsubsection}
    \begin{figure}
        \centering
        %\includegraphics[options]{name}
        \caption{Insérer une image du tracer de la couche limite au voisinage d'une aile d'avion}
    \end{figure}
\end{frame}
\end{document}